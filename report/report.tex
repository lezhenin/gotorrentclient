\documentclass[a4paper,14pt]{extarticle}
\usepackage[utf8]{inputenc}
\usepackage[russian]{babel}
\usepackage{graphicx}
\usepackage{indentfirst}
\usepackage[top=0.8in, bottom=0.8in, left=0.8in, right=0.8in]{geometry}
\usepackage{pgfplots}
\usepackage{amsmath}
\usepackage{setspace}
\usepackage{titlesec}
\usepackage{pdfpages}
\usepackage{subcaption}
\usepackage{float}
\usepackage{longtable}
\usepackage{chngcntr}
\usepackage{pgfplots}
\usepackage{amsfonts}
\usepackage{hhline}
\usepackage{pgfplotstable}
\usepackage{multirow}
\usepackage{tikz,xcolor}
\usepackage{listings}
\usepackage{url}
\usepackage{mmap} 
\usepackage{enumitem}


% \titleformat{\section}[hang]
% {\bfseries}
% {}
% {0em}
% {\hspace{-0.4pt}\large \thesection\hspace{0.6em}}
% 
% 
% \titleformat{\subsection}[hang]
% {\bfseries}
% {}
% {0em}
% {\hspace{-0.4pt}\large \thesubsection\hspace{0.6em}}
% 
% \titleformat{\subsubsection}[hang]
% {\bfseries}
% {}
% {0em}
% {\hspace{-0.4pt}\large \thesubsubsection\hspace{0.6em}}

%\linespread{1.3} % полуторный интервал
%\renewcommand{\rmdefault}{ftm} % Times New Roman

\counterwithin{figure}{section}
\counterwithin{equation}{section}
\counterwithin{table}{section}

\begin{document}
    \begin{titlepage}
        \centering
        Санкт-Петербургский политехнический университет Петра Великого \\
        Институт компьютерных наук и технологий \\
        Кафедра компьютерных систем и программных технологий \\
        \vspace{5.5cm}

        {\centering \textbf{Отчет по лабораторной работе} \\
        \vspace{0.15cm}
        \textbf{Дисциплина}: Тестирование программного обеспечения  \\
        \vspace{0.15cm}
        \textbf{Тема}: Тестирование клиента BitTorrent v1.} \\

        \vspace{5.5cm}

        \begin{table}[H]
            \begin{tabular}{p{\textwidth}@{}r}
            {Выполнил студент гр. 43501/3} \hfill
            \vspace{0.2cm}
            Леженин Ю.И. \\ \vspace{0.2cm}
            Преподаватель \hfill
            \vspace{0.2cm}
            Ахин М.Х. \\
            \vspace{0.4cm}
            {} \hfill { <<\underline{\hspace{0.08\textwidth}}>> \underline{\hspace{0.2\textwidth}}2019 г.} \\
            \end{tabular}
        \end{table}
        \vfill
        {\centering Санкт-Петербург \\
        \vspace{0.15cm}
        2019}
    \end{titlepage}
    
\section*{Введение}    

Разрабатываемое приложение включает три основных компонента: библиотека, 
реализующая взаимодействие по протоколу BitTorrent, интерфейс для 
командной строки, графический интерфейс на базе \texttt{gtk}. Для разработки 
использовался язык go.

В результате выполнения работы был разработан набор модульных и интеграционных 
тестов для библиотки. Для обеспечения качества пользовательского интерфейса 
использовалось ручное тестирование. Выявление ошибок и отладка приложения 
осуществлялось с помощью логирования.

Исходный код приложения хранится в репозитории 
GitHub\footnote{\url{https://github.com/lezhenin/gotorrentclient}}. Сборка и 
тестирование выполняется после каждого обновления кода в репозитории с помощью 
системы непрерывной интеграции Travis 
CI\footnote{\url{https://travis-ci.org/lezhenin/gotorrentclient}}.

\section{Логирование}

Для отслеживания работы приложения и поиска ошибок использовалось логирование.
Сообщения разделены на четыре группы по типу описываемого процесса: 
взаимодействие с другими клиентами, взаимодействие с трекером, управление 
загрузкой и остальное. Каждому типу сообщений можно установить выходной поток 
или файл. Также сообщения разделены по уровням: ошибка, предупреждение, 
информация, отладка и отслеживание. Для реализации логирования использовалась 
библиотека \texttt{logrus}.

\section{Модульное тестирование}\label{sec:unit-test}
    
Приложение содержит шесть основных модулей. Для каждого модуля был разработан 
набор тестов. Тесты можно разделить на три группы: 
\begin{enumerate}[nosep]
 \item Проверка обработки корректных данных.
 \item Проверка обработки некорректных данных.
 \item Проверка обработки исключительных ситуаций.
\end{enumerate}
В тестах первой группы выходные данные сравниваются с ожидаемым результатом. В 
тестах второй группы проверяется реакция программы, в случаях когда входные 
данные имеют неправильную структуру или содержат внутренние противоречия. 
Например, когда длина пакета не соответствует значению, указанному в заголовке. 
К исключительным ситуациям можно отнести разрыв соединения или ожидание ответа 
дольше установленного таймаута.

Длительные тесты, содержащие временные задержки для проверки осуществления 
регулярных событий и таймаутов, были вынесены в отдельную группы и пропускались 
при тестировании в укороченном режиме.

Для написания автоматических тестов использовался пакет стандартной библиотеки  
\texttt{testing} и сторонняя библиотека \texttt{testify}, которая 
предоставляет набор удобных функций для проверки состояния программы в ходе 
тестирования. 

Запуск тестов и вычисление покрытия осуществлялось с помощью утилиты \texttt{go 
test}. Данная утилита поддерживает три типа покрытия: \texttt{set}, 
\texttt{count}, \texttt{atomic}. В первом случае, для каждого оператора 
программы определяется был ли он выполнен. Во втором случае -- считается 
количество выполнений каждого оператора. Тип \texttt{atomic} используется 
для корректного подсчета выполнений операторов в многопоточном приложении. На 
данный момент покрытие типа \texttt{set} операторов разрабатываемой библиотеки 
составляет 87.6\%. 

Запуск тестов происходит в виртуальной машине Travic CI после каждого 
обновления кода в репозитории. Использование Docker контейнера позволило 
упростить настройку системы непрерывной интеграции.
   
\section{Интеграционное тестирование}\label{sec:int-test}

В ходе тестирования выполняется проверка взаимодействия разрабатываемого 
приложения с другими BitTorrent клиентами и трекером. Для этого используются 
соответствующие реализации, разработанные в рамках проекта WebTorrent. 

Создание и запуск тестовой среды осуществляется с помощью средства 
контейнеризации Docker. В ходе тестирования используются три образа: 
 трекер WebTorrent, клиент WebTorrent  и тестируемый клиент. В начале каждого 
теста создается необходимый набор контейнеров, находящихся в одной виртуальной 
сети. Далее в зависимости от реализуемого сценария в контейнерах запускаются 
необходимые команды. 

Созданные тесты на данный момент моделируют следующие ситуации: загрузка данных 
от одного клиента, загрузка данных от двух клиентов, раздача данных одному 
клиенту, раздача данных двум клиентам, одновременная загрузка и раздача данных. 
Если все команды завершились без ошибок и длительность теста не превысила 
установленный таймаут, то тест считается пройденным. 


\section*{Заключение}

Для обеспечения качества разрабатываемого приложения были разработаны набор 
модульных тестов, набор интеграционных тестов и система логирования. Модульное 
тестирование оценивается путем вычисления покрытия операторов программы. 
Окружение для интеграционных тестов разворачивается с помощью средства 
контейнеризации Docker. Для отслеживания ошибок применяется система непрерывной 
интеграции Travis CI. 

В ходе тестирования было обнаружено множество ошибок. Наиболее значимой и 
сложнообнаружимой ошибкой оказалось некорректное управление загрузкой в случае, 
когда размер данных кратен размеру сегмента. Ситуация усложнялась наличием 
особенности в реализации WebTorrent клиента, в результате которой игнорировался 
некорректный запрос и загрузка завершалась успешно.


\end{document}
